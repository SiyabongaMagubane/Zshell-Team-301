\documentclass{article}
\usepackage[margin=2cm]{geometry}
\usepackage{graphicx}
\title{NavUP: Architectural Requirements Specifications and Design}
\author{	 
	Greeff, Claude\\
	\texttt{u13153740@tuks.co.za}
	\and	
}
\begin{document}
	\maketitle
	\section{Architecture Requirements}
		\subsection{Quality Requirements}
		 To create a safe working system that will meet all the required specifications of the NavUP project, specific system requirements need to be looked at in depth. The following are the system requirements that need to be considered. Security, reliability, efficiency and maintainability.
		 \subsubsection{Security}
		 Security is concerned with unauthorized access to specific software functions. An important security measure that needs to be satisfied by the NavUP system is the confidentiality of different user level usernames and passwords. Personal usernames will be set by each user, or their student/staff number will be used to log in. Each user will be required to set their own password. Passwords will be required to be more than 7 characters in length to decrease the likely hood of potential hackers hacking passwords successfully. In the case of a forgotten password, a users password may be reset via a link and an email with a One Time Password. If a user has entered the incorrect password more than 3 times, the account should be blocked and the user should be notified for a password reset. Passwords and any related personal information will be encrypted and safely stored in a 'Users' database.
		 
		  Specific locations will be linked to each user profile. Security will be put in place to allow only that user or other users whom the location is shared with to see the pinned location. Access to the NavUP system will only be made available through the University of Pretoria's firewall network. The firewall will hold as a barrier to deny any outside activity. As different levels of user exist each user will have a specific degree of authority. Admin users will be able to add and remove locations, venues and more permanent locations that will effect the map of the NavUP database. Admin users will have the power to remove lower priority users unwanted actions.
		 
		\subsection{Reliability}
		Concerned with the level of risk and chance of application failure. To increase reliability, downtime needs to be reduced and prevented as well as application errors affecting users. Users will need to be navigated to specific locations for classes, practicals and crucial meetings. User authority needs to be kept in place, not allowing users access to prohibited functions allowing them to make unwanted changes to the NavUP system. The NavUP system will need the ability to recover from a failed state, bring back the system to full operation. Finally the NavUP application should withstand any environmental threats that have the potential to cause system failure.
		 
		\subsection{Efficiency}
		
		\subsection{Maintainability}
	
	\section{Architecture Constraints}
		 
		  
		 
\end{document}